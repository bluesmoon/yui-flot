\documentclass{beamer}

\mode<presentation>
{
  \usetheme{Warsaw}

  \setbeamercovered{transparent}

}

\beamertemplatenavigationsymbolsempty

\setbeamercolor{title in head/foot}{bg=orange!90!black,fg=white}
\setbeamercolor{subsection in head/foot}{bg=black!95,fg=white}
\setbeamercolor{structure}{fg=black}

\setbeamertemplate{blocks}[rounded][shadow=false]

\setbeamertemplate{footline}%
{%
  \leavevmode%
  \hbox{\begin{beamercolorbox}[wd=.5\paperwidth,ht=2.5ex,dp=1.125ex,leftskip=.3cm plus1fill,rightskip=.3cm]{author in head/foot}%
    \usebeamerfont{author in head/foot}\insertdate%
  \end{beamercolorbox}%
  \begin{beamercolorbox}[wd=.5\paperwidth,ht=2.5ex,dp=1.125ex,leftskip=.3cm,rightskip=.3cm plus1fil]{title in head/foot}%
    \usebeamerfont{title in head/foot}\insertshorttitle%
  \end{beamercolorbox}}%
  \vskip0pt%
}%

\usepackage[english]{babel}

\usepackage[latin1]{inputenc}

\usepackage{times}
\usepackage[T1]{fontenc}

\usepackage{ulem}

\author{Philip Tellis / \texttt{philip@bluesmoon.info}}

\title{Javascript charting with YUI-Flot}

\date{\href{http://fosdem.org/2010/}{FOSDEM 2010} / 2010-02-07}


\pgfdeclareimage[height=0.5cm]{bluesmoon-logo}{../ui/default/bodybg}
\pgfdeclareimage[height=11pt]{cc-licence}{../by-nc-sa-3.0-88x31}
\logo{\pgfuseimage{cc-licence}}





\begin{document}

\begin{frame}
  \titlepage
\end{frame}


\section{Introduction}
\subsection{Who, what, why?}

\setbeamertemplate{background}{
  \parbox[c][\paperheight]{\paperwidth}{
    \hfill \pgfimage[height=\paperheight]{../bluesmoon}
  }
}
\begin{frame}{\textit{\$ finger philip}}
  \begin{itemize}
  \item Philip Tellis
  \item \small{\texttt{philip@bluesmoon.info}}
  \item \href{http://bluesmoon.info/}{bluesmoon.info}
  \item @bluesmoon
  \item yahoo
  \item geek
  \end{itemize}
\end{frame}
\setbeamertemplate{background}{}

\begin{frame}{What's this talk all about?}
  \begin{itemize}
  \item Charting options for web developers
  \item Javascript charting with YUI-Flot
  \item Things I learnt
  \item How you can contribute
  \end{itemize}
\end{frame}

\begin{frame}{Why do we need charting on the web?}
  \begin{itemize}
  \item There's so much data on the web
  \item<2-> It's easy for users to access your data over the web
  \item<3-> It's a little more portable than emailing Excel spreadsheets
  \end{itemize}
\end{frame}

\subsection{Charting on the web}

\begin{frame}{Alternatives for charting on the web}
  \begin{itemize}
  \item Static images generated on the server
  \item Flash based dynamic charts with lots of interactivity
  \item Javascript + Canvas based charts with some interactivity
  \item Progressively enhanced CSS only charts
  \item Non-semantic, HTML Table only charts (don't do this)
  \end{itemize}
\end{frame}

\begin{frame}{Image based charts}
  \begin{itemize}
  \item Many charting libraries available
  \item Works on all browsers (kind of)
  \item Static images can be downloaded and pasted into a report
  \item Cannot interact with the data, or easily drill-down
  \end{itemize}
\end{frame}

\begin{frame}{Flash based charts}
  \begin{itemize}
  \item Quite pretty and flashy
  \item Works across most GUI based browsers (with a plugin)
  \item Does not work on the iPhone
  \item Grabs focus - still a problem on linux
  \item Fast, but JS-Flash bridge is slow
  \item Can progressively enhance a data table
  \end{itemize}
\end{frame}

\begin{frame}{Javascript+Canvas based charts}
  \begin{itemize}
  \item Easy for any Javascript developer to hack
  \item Based on Open standards
  \item Does not natively work with Internet Explorer
  \item Easy to integrate with other javascript libraries
  \item Can progressively enhance a data table
  \end{itemize}
\end{frame}

\begin{frame}{CSS only charts}
  \begin{itemize}
  \item \href{http://icant.co.uk/csscharts/}{http://icant.co.uk/csscharts/}
  \item Very simple code
  \item Data needs to be encoded twice
  \item Only vertical bar charts
  \end{itemize}
\end{frame}

\subsection{YUI-Flot}

\begin{frame}{What is flot?}
  \begin{itemize}
  \item A pure javascript charting library
  \item Uses the HTML Canvas tag for drawing
  \item Really smooth, and very pretty
  \item Uses \texttt{excanvas} to work on IE
  \end{itemize}
\end{frame}

\begin{frame}{Why I like it}
  \begin{itemize}
  \item Looks good -- very important based on your target audience
  \item Good event model
  \item Fast at drawing and redrawing
  \item Ability to ``zoom in'' to sections of the data
  \end{itemize}
\end{frame}

\begin{frame}{Why I ported it to YUI}
  \begin{itemize}
  \item My Javascript library of choice is YUI
  \item My project required a charting widget
  \item Flash based charts caused me problems on linux
  \item Flot seems to have a good community around it
  \item Most important -- I already knew and was using Javascript
  \end{itemize}
\end{frame}

\section{YUI-Flot}
\subsection{What can it do?}

\setbeamertemplate{background}{
  \parbox[c][\paperheight]{\paperwidth}{
    \pgfimage[height=\paperheight]{./scatter} \hfill 
  }
}
\begin{frame}{Draw basic x/y scatter plots}
  \begin{columns}[c]
  \column{.6\textwidth}
    \begin{block}{}
      \small{Random set of x,y points} \\
      \texttt{\tiny{[[10, 20], [15, 13], [8, 32], [12, 7], [17, 19], [23, 8], [3, 5]]}} \\
      \href{http://bluesmoon.github.com/yui-flot/yui-examples/scatter.html}{\small{yui-examples/scatter.html}} \\
    \end{block}
  \column{.4\textwidth}
  \end{columns}
\end{frame}

\setbeamertemplate{background}{
  \parbox[c][\paperheight]{\paperwidth}{
    \pgfimage[height=\paperheight]{./timeseries} \hfill 
  }
}
\begin{frame}{Series plots}
  \begin{columns}[c]
  \column{.3\textwidth}
  \column{.7\textwidth}
    \begin{block}{}
      \small{X-axis contains Date objects or timestamps} \\
      \small{Resolution is in seconds} \\
      \small{strftime formatting} \\
      \href{http://bluesmoon.github.com/yui-flot/yui-examples/timeseries.html}{\small{yui-examples/timeseries.html}} \\
    \end{block}
  \end{columns}
\end{frame}

\setbeamertemplate{background}{
  \parbox[c][\paperheight]{\paperwidth}{
    \pgfimage[height=\paperheight]{./multiseries} \hfill 
  }
}
\begin{frame}{Multiple series}
  \begin{columns}[c]
  \column{.3\textwidth}
  \column{.7\textwidth}
    \begin{block}{}
      \small{3 series with a common X value} \\
      \small{Schema style data structure -- out of a DB}
      \href{http://bluesmoon.github.com/yui-flot/yui-examples/multiseries.html}{\small{yui-examples/multiseries.html}} \\
    \end{block}
  \end{columns}
\end{frame}
\setbeamertemplate{background}{}

\subsection{How to use it?}
% 1. Include code into your HTML
% 2. Get your data from somewhere
% 3. Instantiate the chart

\subsection{Differences from Flot}
% Based of flot 0.5, flot is now at 0.6
% locale support for time series
% timestamps are in seconds and not in milliseconds
% More than two axes (scales) in each direction
% Axis labels - different from tick labels
% Log scale for all axes (also in Flot 0.6)
% Full strftime based date formatter
% Point dropping

\section{Enlightenment}
\subsection{Future}
% Move up to flot 0.6
% I like the plugin model in flot 0.6
% Add error bars & candlesticks
% Possibly add a smoothing plugin
% Upgrade to YUI 3
% Move it into the YUI gallery

\subsection{Things I learnt}
% Objects in javascript are passed in by reference
% There's more to plotting a chart than just drawing lines
% You can plot more than two dimensions on a 2d canvas
% You can rotate HTML elements using CSS or SVG

\subsection{How to contribute}
% Fork the project on github
% Sign the YUI CLA and get code back in

\section{}
\begin{frame}{\textit{Thanks, Merci, Bedankt, Danke Sch\"on}}
  \begin{itemize}
  \item Jochen Maes \& Philip Paeps
  \item Ole Laursen \& IOLA
  \item Yahoo! Developer Network
  \item The YUI team
  \end{itemize}
\end{frame}

\setbeamertemplate{background}{
  \parbox[c][\paperheight]{\paperwidth}{
    \hfill \pgfimage[height=\paperheight]{../bluesmoon}
  }
}
\begin{frame}{\textit{contact me}}
  \begin{itemize}
  \item Philip Tellis
  \item \small{\texttt{philip@bluesmoon.info}}
  \item \href{http://bluesmoon.info/}{bluesmoon.info}
  \item @bluesmoon
  \item yahoo
  \item geek
  \end{itemize}
\end{frame}
\setbeamertemplate{background}{}


\end{document}


